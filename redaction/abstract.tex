\section*{Abstract}

%intro : interet, problematique, contexte 
This study is about sound propagation in periodic structures, focusing on the case of a waveguide, periodically loaded by Helmholtz resonators, taking in account viscothermal losses (distributed in the duct and localised at each resonator). Research on resonant acoustic meta-materials is closely linked to works on lattice in the fields of electromagnetism and crystallography.\\~\\ %Historically, research on lattice has been more developed in this two field than in acoustics%historiquement, c'est ces deux derniers domaines dans lequel ça a été développé, donc moins en acoustique


This kind of structures is used for its ability to create a dispersive medium. Indeed, some frequencies are absorbed for two reasons. One reason is that resonators absorb the energy at the frequencies near their resonances. Other frequency gaps, called Bragg gaps, are created by the periodicity of the structure that localises waves for witch the wave length is twice the distance between two resonators.\\~\\

Theoretical works and numerical simulations are leaded in this report in order to describe the frequency response for finite, infinite, disordered and ordered lattice. 

Then, using the dispersion relation of the system and introducing a defect on the cavity length of one resonator, a phenomenon of localised wave is studied numerically and observed experimentally. For this purpose, the choice of the resonance frequency of the defect and its position in the lattice is carefully considered. If the defect resonate in a frequency gap, the pressure wave resulting from this resonance cannot propagate and is restricted in the vicinity of the defect.\\~\\ 

%So what ? Conclusion ? Parler du banc de mesures, des soucis ? 

%contribution aux recherche en non-lineaire.
Keywords: lossy periodic structures, dispersion, localised mode 


%methodologie : comment

%résultats principaux, conclusion, implication

