\chapter*{Annexes}

\section*{Annexe 1: Ajout des pertes viscothermique dans les équations}


\section*{Annexe 2: Paramètres du réseau étudié }
La liste des dimensions du réseau utilisé lors des expériences et des simulations est la suivante:
\begin{itemize}
\item Le guide a un rayon de $R_t = 2.5~cm$ et une épaisseur de $ep = 0.5~cm$.
\item Les résonateur sont composées de 2 tubes: le col de $R_n = 2~cm$ de rayon et $L_n = 2~cm$ de longueur, la cavité de $R_c = 2.15~cm$ de rayon et de longueur variable $L_c$. C'est cette dernière longueur qui permet de faire varier la fréquence de résonance du résonateur. 
\end{itemize}

\bigskip
Les corrections apportées aux cols des résonateurs sont les suivantes:
\begin{eqnarray*}
l_1 & = &  0.82 \left[ 1 - 1.35 \frac{R_n}{R_c} + 0.31 \left(\frac{R_n}{R_c} \right)^3  \right] R_n \\
l_2 & = &  0.82 \left[ 1 - 0.235 \frac{R_n}{R_t} - 1.32 \left( \frac{R_n}{R_t} \right)^2 +1.54 \left( \frac{R_n}{R_t}\right)^3 - 0.86 \left( \frac{R_n}{R_t}\right)^4  \right] R_n \\
L_{corr} & = &  l_1 + l_2
\end{eqnarray*}
