\chapter{Conclusion}

Ce projet a permis l'étude d'un réseau périodique de résonateurs de Helmholtz en prenant en compte les pertes visco-thermiques. 

Dans un premier temps, un réseau périodique infini a été étudié afin de se familiariser avec le formalisme matriciel (onde de Bloch): celui-ci est très apte à décrire ce genre de système. Les simulations faites sur la base de ce modèle théorique ont été en parfait accord avec des mesures effectuées sur le banc d'expérience du LAUM\footnote{\samepage Laboratoire d'Acoustique de l'Université du Maine}. Une étude succincte d'un faible désordre dans le réseau a également été abordée : un désordre sur la position des résonateur affecte les bandes de Bragg, tandis qu'un désordre sur la longueur des cavités modifie les bandes interdites liées à l'impédance des résonateurs.

 
Dans un second temps, un défaut a été ajouté dans le réseau afin de générer un mode localisé. Là encore, les simulations sont en accord avec les expériences. Bien que le phénomène soit difficile à observer du fait de la localisation de la pression dans le tube à cause des bandes interdites, les mesures sur un réseau avec un faible nombre de résonateurs ont permis de mettre en évidence l’existence d'un mode de défaut sur les coefficients de transmission et réflexion et de connaître la fréquence exacte d’excitation du mode. Il a alors été possible de mesurer la pression dans le tube en excitant près du défaut à cette fréquence. La différence d'ordre de grandeur des pressions entre un réseau avec et sans défaut s'est montrée flagrante et a corroboré les expériences précédentes, montrant qu'il s'agissait bien d'un mode de défaut.

Ces travaux peuvent avoir pour suite l'ajout de phénomènes non-linéaires afin de créer des filtres intéressant à appliquer dans la création de méta-matériaux. Le résonateur de Helmholtz n'étant pas linéaire pour de grandes amplitudes, sa fréquence de résonance s'abaisse quand l'amplitude augmente (pour de grandes amplitudes). Plusieurs applications peuvent donc découler de ce phénomène.

Par exemple, dans le cas d'un résonateur dont la fréquence est dans une bande interdite en amplitude faible, cette résonance pourra ne plus être dans une bande interdite pour une amplitude plus forte : un filtrage en amplitude est alors possible (filtrage dynamique).


De plus, une génération d'harmoniques supérieurs par ce même résonateur est possible en forte amplitude. Ainsi, si une source excite ce défaut, engendrant des harmoniques supérieurs qui ne sont pas dans une bande interdite, la propagation de ces harmoniques est possible. Mais si ce résonateur est placé loin de la source, l'amplitude d'excitation au niveau du défaut sera très faible, et les harmoniques ne seront pas générés. Il est donc possible de créer des système asymétrique sur ce principe.







