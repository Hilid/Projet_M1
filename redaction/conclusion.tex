\chapter{Conclusion}

Finalement, ce projet à permis l'étude d'un réseau périodique de résonateurs de Helmholtz en prenant en compte les pertes visco-thermiques. 

Tout d'abord, un réseau périodique infini a été étudié afin de ce familiariser avec le formalisme matriciel (onde de Blocj): celui-ci semble le plus apte a décrire ce genre de système. Les simulations faites sur la base de ce modèle théorique ont été en parfaite accords avec des mesures effectués sur le banc d'expérience du LAUM. C'est dans cette partie qu'une étude succincte d'un faible désordre dans le réseau a été abordé.

Dans un second temps, un  défaut a été ajouté dans le réseau afin de générer un mode localisé. La encore les simulations créée sont en accord avec les expériences. Bien que le phénomène soit difficile à observer du fait de la localisation de la pression dans le tube à cause des bandes interdites, les mesures sur un réseau avec un faible nombre de résonateurs en permis de mettre en évidence l’existence d'un mode de défaut sur les coefficient de transmission et réflexions et d'en connaître l'exact fréquence. Il a alors été possible de mesurer la pression dans le tube en insérant un microphone dans celui-ci et en excitant près du défaut à la bonne fréquence. La différence d'ordre de grandeur des pressions entre un réseau avec et sans défaut c'est montrée criante et a corroboré les expériences précédentes en montrant qu'il s'agissait bien d'un mode de défaut.

Ces travaux ont pour suite logique l'ajout de phénomènes non-linéaires afin de créer des filtres intéressant à appliquer dans la création de méta-matériaux. On peut par exemple imaginer un système avec un défaut en début de réseau: si l'onde incidente de forte amplitude excite le défaut, il y a propagation d'harmonique supérieurs dans le réseau et celle-ci ne sont pas dans une bande interdite, si l'onde incidente se trouve de l'autre côté du réseau la propagation est impossible. On a donc affaire à un système asymétrique, une sorte de diode acoustique. Ce genre de système peux aussi faire office de filtre en amplitude en ne laissant passer qu'un signal de fort niveau: des filtrages dynamique peuvent donc être conçus sur la base de modes localisés identiques à ceux étudiés ici en prenant en compte les non-linéarités du défaut.


