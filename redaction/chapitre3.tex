\chapter{Visualisation expérimental d'un mode localisé}
Différents problèmes rendent la mesure d'un mode localisé difficile.

\begin{itemize}
\item Tout d'abord il n'est pas trivial de trouver une configuration qui permette la génération d'un mode localisé car c'est un phénomène difficile à visualiser et ce même en simulation.

\item Le deuxième problème réside dans le fait que la propagation de l'onde dans le réseau soit difficile: la bande interdite rentre la décroissance de l'onde exponentielle. Par conséquence, exciter le défaut est difficile.

\item Enfin le dernier problème ce situe au niveau des mesures. Comme le champ de pression du mode est localisé, il n’apparaît pas sur les coefficients de transmission et réflexion du réseau complet (ou alors très peu si le défaut se trouve sur un des bords du réseau).
\end{itemize}


\section{Protocole expérimental}
\section{Résultat et comparaison avec la simulation}
