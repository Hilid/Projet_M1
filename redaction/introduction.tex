\addcontentsline{toc}{section}{\textbf{Introduction}}
\chapter*{Introduction}


La propagation dans les réseaux n'est pas un sujet nouveau : celui-ci a déjà été abordé en profondeur notamment dans des recherches en cristallographie et en électromagnétisme. Les phénomènes liés à la propagation dans les réseaux sont cependant moins bien connus en acoustique, notamment en ce qui concerne la propagation accompagnée de non-linéarités.\\~\\


Ce travail s'inscrit dans le cadre de recherches sur les méta-matériaux composés de structures résonantes périodiques. Ce genre de matériaux dispose de propriétés uniques telles que la réfraction négative ou encore la super-absorption. Notre projet consiste à mieux comprendre le phénomène de mode localisé et à l'observer expérimentalement et ce, en lien avec des chercheurs qui pourront exploiter les résultats obtenus, en y ajoutant des phénomènes d'acoustique non-linéaire.\\~\\


L'étude de la propagation acoustique dans les réseaux est un domaine complexe dès lors que les éléments constitutif du réseau disposent de leurs propres fréquences de résonance. En effet, deux types de bandes interdites (bandes de fréquences pour lesquelles la transmission est nulle) sont alors présentes et peuvent interagir : les bandes de Bragg, liées à la géométrie du réseau, et les bandes interdites liées à l'impédance des éléments du réseau. De plus, si un défaut est ajouté dans le réseau, il peut alors se produire un phénomène de localisation de la pression dans le tube : un mode localisé (ou mode de défaut) peut alors être observé. \\~\\


Le but du projet est donc d'étudier la propagation acoustique dans un réseau composé de résonateurs de Helmholtz, en prenant en compte les pertes visco-thermiques intrinsèques à la propagation dans ce type de structure. Une fois la théorie sur les réseaux périodiques développée, une étude sur l'ajout d'un défaut dans le réseau est menée. Pour cela, des simulations numériques sont confrontées à des mesures sur un réseau se trouvant au Laboratoire d'Acoustique de l'Université du Maine. Toutes les simulations ainsi que les données des expériences sont disponibles à l'adresse internet suivante : \url{https://github.com/Hilid/Projet_M1}.
