\chapter*{Introduction}


La propagation dans les réseaux n'est pas un sujet nouveau: celui-ci a déjà été abordé en profondeur notamment dans des recherches en cristallographie et en électromagnétisme. Cependant sont pendant acoustique à fait l'objet de peut de recherches.

L'étude de la propagation acoustique dans les réseaux est un domaine complexe des lors que les éléments constitutif du réseau disposent de leurs propres fréquences de résonance. En effet, 2 types de bandes interdites sont alors présentes et peuvent interagir: les bandes de Bragg lié à la géométrie du réseau et les bandes interdites liés au comportement fréquentiel des éléments du réseau. Si on défaut est ajouté dans le réseau, il peut alors ce produire un phénomène de localisation de la pression dans le tube: c'est ce qu'on appelle un mode localisé (ou mode de défaut).

Ce travail s'inscrit dans le cadre de recherche sur les méta-matériaux composés de structures résonantes localisés. Ce genre de matériaux dispose de propriétés uniques tels que la réfraction négative ou encore des matériaux super-absorbant. Notre projet consiste à mieux comprendre le phénomène de mode localisé afin que des chercheurs puissent exploiter les résultats obtenues en y ajoutant des phénomènes d'acoustique non-linéaire.

\bigskip
Le but du projet est donc d'étudier la propagation dans un réseau composés de résonateurs de Helmholtz en prenant en compte les pertes visco-thermique intrinsèque à la propagation dans ce type de structure. Une fois la théorie sur les réseaux périodiques assimilé, une étude sur l'ajout d'un défaut dans le réseau sera effectuée. Des simulations seront confrontés à des mesures sur un réseau se trouvant au laboratoire d'acoustique de l'université du Maine.

%\begin{itemize}
%
%\item domaine de recherche: méta-matériaux, filtrage analogique, tout les phénomènes de propagation.
%\item Citation des sources => optique puis electromagnetique puis acoustique
%\item but: comprendre propagation dans réseau et visualisation expérimental de mode localisé
%\item plan: Approche théorique de la propa dans réseau ac, ajout de singularité pour mode localisé, expérimentation
%\end{itemize}