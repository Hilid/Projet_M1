\chapter{Ajout d'une singularité dans le réseau}
La seconde partie de ce projet est une étude portant sur l'observation d'un mode localisé dans le réseau précédemment étudié. Pour cela, un résonateur est modifié de façon à créer une singularité dans le réseau. \\
Le banc de manipulation ne nous permet de modifier que la longueur des cavités des résonateurs et non leur position. C'est donc en changeant ce paramètre que sera introduite et étudiée la singularité. La longueur de la cavité singulière est notée $L_{c_{s}}$.\\

En fonction du choix de la longueur de la cavité singulière -et donc de la fréquence de résonance du résonateur associé- différents phénomènes peuvent être observés. 


\section{Cas d'une singularité hors bande de Bragg}


\section{Cas d'une singularité dans la bande de Bragg}

\section{Calcul numérique de la pression dans le réseau}

\section{Étude de l'influence de la position du défaut}